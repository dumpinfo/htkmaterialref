%/* ----------------------------------------------------------- */


%/*                                                             */


%/*                          ___                                */


%/*                       |_| | |_/   SPEECH                    */


%/*                       | | | | \   RECOGNITION               */


%/*                       =========   SOFTWARE                  */ 


%/*                                                             */


%/*                                                             */


%/* ----------------------------------------------------------- */


%/*         Copyright: Microsoft Corporation                    */


%/*          1995-2000 Redmond, Washington USA                  */


%/*                    http://www.microsoft.com                */


%/*                                                             */


%/*   Use of this software is governed by a License Agreement   */


%/*    ** See the file License for the Conditions of Use  **    */


%/*    **     This banner notice must not be removed      **    */


%/*                                                             */


%/* ----------------------------------------------------------- */


%


% HTKBook - Steve Young 15/11/95


%





\mychap{Discrete and Tied-Mixture Models}{discmods}





\sidepic{Tool.disc}{80}{ 


Most of the discussion so far has focussed on using \HTK\


to model sequences of continuous-valued vectors.  In contrast, this chapter


is mainly concerned with using \HTK\ to model sequences of discrete


symbols.  


Discrete symbols arise naturally in modelling many types of


data, for example, letters and words, bitmap images, and DNA sequences.


Continuous signals can also be converted to discrete symbol sequences


by using a quantiser and in particular, speech vectors can be


\textit{vector quantised} as described in section~\ref{s:vquant}.


In all cases, \HTK\ expects a set of $N$ discrete symbols to be represented by


the contiguous sequence of integer numbers from 1 to $N$.


}\index{discrete HMMs}





In \HTK\ discrete probabilities are regarded as being closely analogous


to the mixture weights of a continuous density system.  As a consequence,


the representation and processing of discrete HMMs shares a great deal


with continuous density models.  It follows from this that most of the


principles and practice developed already are equally applicable to


discrete systems.  As a consequence, this chapter can be quite brief.





The first topic covered concerns building HMMs for discrete


symbol sequences.  The use of discrete HMMs with speech is then


presented.  The tool \htool{HQuant} is described and the method


of converting continuous speech vectors to discrete symbols is


reviewed.  This is 


followed by a brief discussion of tied-mixture systems which can be 


regarded as a compromise between continuous and discrete density systems.


Finally, the use of the \HTK\ tool \htool{HSmooth} for


parameter smoothing by deleted interpolation is presented.





\mysect{Modelling Discrete Sequences}{discseq}





Building HMMs for discrete symbol sequences is essentially the same


as described previously for continuous density systems. 


Firstly, a prototype HMM definition must be specified in order


to fix the model topology.  For example, the following


is a 3 state ergodic HMM in which the emitting states


are fully connected.


\begin{verbatim}


    ~o <DISCRETE> <StreamInfo> 1 1 


    ~h "dproto"


    <BeginHMM>


       <NumStates> 5 


       <State> 2 <NumMixes> 10


          <DProb> 5461*10


       <State> 3 <NumMixes> 10


          <DProb> 5461*10


       <State> 4 <NumMixes> 10


          <DProb> 5461*10


       <TransP> 5


           0.0 1.0 0.0 0.0 0.0


           0.0 0.3 0.3 0.3 0.1


           0.0 0.3 0.3 0.3 0.1


           0.0 0.3 0.3 0.3 0.1


           0.0 0.0 0.0 0.0 0.0


    <EndHMM>


\end{verbatim}


As described in chapter~\ref{c:HMMDefs}, the notation for discrete


HMMs borrows heavily on that used for continuous density models


by equating mixture components with symbol indices.  Thus,


this definition assumes that each training data sequence contains


a single stream of symbols indexed from 1 to 10.  In this example,


all symbols in each state have been set to be equally likely\footnote{


Remember that discrete probabilities are scaled such that


32767 is equivalent to a probability of 0.000001 and 0 is 


equivalent to a probability of 1.0


}.  If prior information is available then this can of course be used


to set these initial values.





The training data needed to build a discrete HMM can take one of two forms. It


can either be discrete (\texttt{SOURCEKIND=DISCRETE}) in which case it consists


of a sequence of 2-byte integer symbol indices.  Alternatively, it can consist


of continuous parameter vectors with an associated VQ codebook.  This latter


case is dealt with in the next section.  Here it will be assumed that the data


is symbolic and that it is therefore stored in discrete form.\index{discrete


data} Given a set of training files listed in the script file


\texttt{train.scp}, an initial HMM could be estimated using


\begin{verbatim}


    HInit -T 1 -w 1.0 -o dhmm -S train.scp -M hmm0 dproto


\end{verbatim}


This use of \htool{HInit} is identical to that which would be


used for building whole word HMMs where no associated label file is


assumed and the whole of each training sequence is used to estimate


the HMM parameters.  Its effect is to read in the prototype


stored in the file \texttt{dproto} and then use the training examples


to estimate initial values for the output distributions


and transition probabilities.  This is done by firstly uniformly 


segmenting the data and for each segment counting the number of occurrences


of each symbol.  These counts are then normalised to provide output distributions


for each state.  \htool{HInit} then uses the Viterbi algorithm to resegment


the data and recompute the parameters.  This is repeated until convergence


is achieved or an upper limit on the iteration count is reached.


The transition probabilities at each step are estimated simply by


counting the number of times that each transition is made in the Viterbi alignments


and normalising.  The final model is renamed \texttt{dhmm} and stored in


the directory \texttt{hmm0}.





When building discrete HMMs, it is important to floor the discrete


probabilites so that no symbol has a zero probability.  This is 


achieved using the \texttt{-w} option which specifies a floor value


as a multiple of a global constant called \texttt{MINMIX} whose


value is $10^{-5}$. 





The initialised HMM created by \htool{HInit} 


can then be further refined if desired by using \htool{HRest}


to perform Baum-Welch re-estimation.  It would be invoked in a similar


way to the above except that there is now no need to rename the model.


For example,


\begin{verbatim}


    HRest -T 1 -w 1.0 -S train.scp -M hmm1 hmm0/dhmm


\end{verbatim}


would read in the model stored in \texttt{hmm0/dhmm} and write out a new


model of the same name to the directory \texttt{hmm1}.





\mysect{Using Discrete Models with Speech}{speechvq}





As noted in section~\ref{s:vquant}, discrete HMMs can be used to model


speech by using a vector quantiser to map continuous density vectors into


discrete symbols.  A vector quantiser depends on a so-called \textit{codebook} 


which defines a set of partitions of the vector space.  Each partition


is represented by the mean value of the speech vectors belonging


to that partition and optionally  a variance representing the spread.


Each incoming speech vector is then


matched with each partition and assigned the index corresponding


to the partition which is closest using a Mahanalobis distance metric.





In \HTK\ such a codebook can be built using the tool \htool{HQuant}.  This tool


takes as input a set of continuous speech vectors, clusters them and uses


the centroid and optionally the variance of each cluster to define


the partitions.  \htool{HQuant} can build both linear and tree structured


codebooks.  To build a linear codebook, all training vectors are initially


placed in one cluster and the mean calculated.  The mean is then perturbed


to give two means and the training vectors are partitioned according to


which mean is nearest to them.  The means are then recalculated and the


data is repartitioned.  At each cycle, the total distortion (i.e. total


distance between the cluster members and the mean) is recorded and repartitioning


continues until there is no significant reduction in distortion.  The whole


process then repeats by perturbing the mean of the cluster with the highest


distortion.  This continues until the required number of clusters have been


found.





Since all training vectors are reallocated at every cycle, this is an


expensive algorithm to compute.  The maximum number of iterations within


any single cluster increment can be limited using the configuration


variable \texttt{MAXCLUSTITER}\index{maxclustiter@\texttt{MAXCLUSTITER}} 


and although this can speed-up the computation


significantly, the overall training process is still computationally expensive.


Once built, vector quantisation is performed by scanning all codebook


entries and finding the nearest entry.  Thus, if a large codebook is used,


the run-time VQ look-up operation can also be expensive.





As an alternative to building a linear codebook, a tree-structured codebook


can be used.  The algorithm for this is essentially the same as above


except that every cluster is split at each stage so that the first cluster


is split into two, they are split into four and so on.  At each stage, the


means are recorded so that when using the codebook for vector quantising


a fast binary search can be used to find the appropriate leaf cluster.


Tree-structured codebooks are much faster to build since there is no


repeated reallocation of vectors and much faster in use since only $O(\log_2 N)$


distance need to be computed where $N$ is the size of the codebook.


Unfortunately, however, tree-structured codebooks will normally incur higher 


VQ distortion for a given codebook size.





When delta and acceleration coefficients are used, it is usually best


to split the data into multiple streams (see section~\ref{s:streams}.


In this case, a separate codebook is built for each stream.





As an example, the following invocation of \htool{HQuant} would


generate a linear codebook in the file \texttt{linvq} using


the data stored in the files listed in \texttt{vq.scp}.  


\begin{verbatim}


   HQuant -C config -s 4 -n 3 64 -n 4 16 -S vq.scp linvq


\end{verbatim}


Here the configuration file \texttt{config} specifies the \texttt{TARGETKIND}


as being \texttt{MFCC\_E\_D\_A} i.e.\ static coefficients plus deltas plus


accelerations plus energy.  The \texttt{-s} options requests that this


parameterisation be split into


4 separate streams.  By default, each individual codebook has 256 entries, however,


the \texttt{-n} option can be used to specify alternative sizes.





If a tree-structured codebook was wanted rather than a linear codebook,


the \texttt{-t} option would be set.


Also the default is to use Euclidean distances both for building the


codebook and for subsequent coding.  Setting the \texttt{-d} option


causes a diagonal covariance Mahalanobis metric to be used and 


the \texttt{-f} option causes a full covariance Mahalanobis metric


to be used.





\index{hcopy@\htool{HCopy}}


\sidefig{vqtohmm}{55}{VQ Processing}{-4}{


Once the codebook is built, normal speech vector files can be 


converted to discrete files using \htool{HCopy}.  


This was explained 


previously in section~\ref{s:vquant}.  The basic mechanism is to


add the qualifier \texttt{\_V} to the 


\texttt{TARGETKIND}.\index{qualifiers!aaav@\texttt{\_V}}  This causes


\htool{HParm} to append a codebook index to each constructed observation


vector.  If the configuration variable \texttt{SAVEASVQ} is set true, then


the output routines in \htool{HParm} will discard the original vectors


and just save the VQ indices in a \texttt{DISCRETE} file. 


Alternatively, \HTK\ will regard any speech vector with \texttt{\_V} set


as being compatible with discrete HMMs.  Thus, it is not necessary


to explicitly create a database of discrete training files if


a set of continuous speech vector parameter files already exists.


Fig.~\href{f:vqtohmm} illustrates this process.


}\index{saveasvq@\texttt{SAVEASVQ}}


\index{targetkind@\texttt{TARGETKIND}}





Once the training data has been configured for discrete HMMs, the 


rest of the training process is similar to that previously described.


The normal sequence is to build a set of monophone models and then


clone them to make triphones.  As in continuous density systems, 


state tying can be used to improve the


robustness of the parameter estimates.  However, in the case of discrete HMMs,


alternative methods based on interpolation are possible.  These are discussed


in section~\ref{s:psmooth}.





\mysect{Tied Mixture Systems}{tiedmix}





\index{tied-mixtures}


Discrete systems have the advantage of low run-time computation.  However,


vector quantisation reduces accuracy and this can lead to poor performance.


As a intermediate between discrete and continuous, a fully tied-mixture


system can be used.


Tied-mixtures are conceptually just another example of the general parameter tying


mechanism built-in to \HTK.  However, to use them effectively in


speech recognition systems a number of storage and computational 


optimisations must be made.  Hence, they are given special treatment in \HTK.





When specific mixtures are tied as in 


\begin{verbatim}


     TI "mix" {*.state[2].mix[1]} 


\end{verbatim}


then a Gaussian mixture component is shared across all of the owners


of the tie.  In this example, all models will share the same Gaussian


for the first mixture component of state 2.  However, if the mixture


component index is missing, then all of the mixture components participating in


the tie are {\it joined} rather than tied. More specifically, the commands


\begin{verbatim}


     JO 128 2.0


     TI "mix" {*.state[2-4].mix} 


\end{verbatim}


has the following effect.  All of the mixture components in states 2 to 4 of


all models are collected into a pool.  If the number of components


in the pool exceeds 128, as set by the preceding join command 


\texttt{JO}\index{jo@\texttt{JO} command}, then


components with the smallest weights are removed until the pool size is exactly


128.  Similarly, if the size of the initial pool is less than 128, then mixture


components are split using the same algorithm as for the Mix-Up \texttt{MU}


command.\index{mixture tying}   All states then share all of the 


mixture components in this pool.  The new mixture weights are chosen to be proportional


to the log probability of the corresponding new mixture component mean with


respect to the original distribution for that state.  The log is used here


to give a wider spread of mixture weights.  All mixture weights are floored


to the value of the second argument of the \texttt{JO} command times 


\texttt{MINMIX}\index{minmix@\texttt{MINMIX}}.





The net effect of the above two commands is to create a set of \texttt{tied-mixture}


HMMs\footnote{Also called {\it semi-continuous} HMMs in the the literature.}


where the same set of mixture components is shared across all states of


all models.  However, the type of the HMM set so created will still be


\texttt{SHARED} and the internal representation will be the same as for


any other set of parameter tyings.   To obtain the optimised representation 


of the tied-mixture weights


described in section~\ref{s:tmix}, the following \htool{HHEd}


\texttt{HK}\index{hk@\texttt{HK} command} command must be issued


\begin{verbatim}


     HK TIEDHS


\end{verbatim}


This will convert the internal representation to the special tied-mixture


form in which all of the tied mixtures are stored in a global table and


referenced implicitly instead


of being referenced explicitly using pointers.





Tied-mixture HMMs work best if the information relating to different sources


such as delta coefficients and energy are separated into distinct data streams.


This can be done by setting up multiple data stream HMMs from the outset.  


However, it is simpler to use the 


\texttt{SS}\index{ss@\texttt{SS} command}


command in \htool{HHEd} to split the data streams of the currently loaded HMM set.


Thus, for example, the command


\begin{verbatim}


     SS 4 


\end{verbatim}


would convert the currently loaded HMMs to use four separate data streams


rather than one.  When used in the construction of tied-mixture HMMs


this is analogous to the use of multiple codebooks in discrete density 


HMMs.





The procedure for building a set of tied-mixture HMMs may be summarised


as follows\index{tied-mixtures!build procedure}


\begin{enumerate}


\item Choose a {\it codebook} size for each data stream and then 


   decide how many Gaussian components will be needed from an initial set of


    monophones to approximately fill this codebook.


    For example, suppose that there are 48 three state monophones.  If


    codebook sizes of 128 are chosen for streams 1 and 2, and


    a codebook size of 64 is chosen for stream 3 then single Gaussian


    monophones would provide enough mixtures in total to fill the codebooks.


\item Train the initial set of monophones.


\item Use \htool{HHEd} to first split the HMMs into the required number of


    data streams, tie


    each individual stream and then convert the tied-mixture HMM set to 


    have the kind \texttt{TIEDHS}.  


    A typical script to do this for four streams would be


\begin{verbatim}


    SS 4


    JO 256 2.0


    TI st1 {*.state[2-4].stream[1].mix}


    JO 128 2.0


    TI st2 {*.state[2-4].stream[2].mix}


    JO 128 2.0


    TI st3 {*.state[2-4].stream[3].mix}


    JO 64 2.0


    TI st4 {*.state[2-4].stream[4].mix}


    HK TIEDHS


\end{verbatim}


\item Re-estimate the models using \htool{HERest} in the normal way.


\end{enumerate}


Once the set of retrained tied-mixture models has been produced, context


dependent models can be constructed using similar methods to those


outlined previously.





When evaluating probabilities in tied-mixture systems, it is often


sufficient to sum just the most likely mixture components since for any


particular input vector, its probability with respect to many of the Gaussian


components will be very low.\index{pruning!in tied mixtures}


\HTK\ tools recognise \texttt{TIEDHS}


HMM sets as being special in the sense that additional optimisations


are possible. When full tied-mixtures are used, then an additional layer of pruning


is applied.  At each time frame, the log probability of the current observation


is computed for each mixture component.  Then only those components which lie within


a threshold of the most likely component are retained.  This 


pruning is controlled by the \texttt{-c} option in 


\htool{HRest}, \htool{HERest} and \htool{HVite}.





\mysect{Parameter Smoothing}{psmooth}





When large sets of context-dependent triphones are built using


discrete models or


tied-mixture models, under-training\index{under-training} can be a 


severe problem since each 


state has a large number of mixture weight parameters to estimate.


The \HTK\ tool \htool{HSmooth} allows these discrete probabilities or


mixture component weights


to be smoothed with the monophone weights using a technique called 


deleted interpolation\index{deleted interpolation}.  





\htool{HSmooth} is used in combination with \htool{HERest}


working in parallel mode.  The training data is split


into blocks and each block is used separately to re-estimate the


HMMs.  However, since \htool{HERest} is in parallel mode, it outputs a dump


file of accumulators instead of updating the models.  \htool{HSmooth} is then


used in place of the second pass of \htool{HERest}.  It reads in the 


accumulator information from each of the blocks, performs deleted


interpolation smoothing on the accumulator values and then outputs


the re-estimated HMMs in the normal way.





\htool{HSmooth}\index{hsmooth@\htool{HSmooth}} implements a 


conventional deleted interpolation scheme.


However, optimisation of the smoothing weights uses a fast 


binary chop\index{binary chop} scheme 


rather than the more usual Baum-Welch approach.


The algorithm for finding the optimal interpolation weights for a given


state and stream is as follows where the description is given in terms


of tied-mixture weights but the same applies to discrete probabilities.  





Assume that \htool{HERest}


has been set-up to output $N$ separate blocks of accumulators.


Let $w_i^{(n)}$ be the $i$'th


mixture weight based on the accumulator blocks $1$ to $N$ but excluding


block $n$, and let $\bar{w}_i^{(n)}$ be the corresponding context


independent weight.  Let $x_i^{(n)}$ be the $i$'th mixture weight 


count for the deleted block $n$.  The derivative of the log


likelihood of the deleted block, given the probability distribution with


weights $c_i = \lambda w_i + (1 - \lambda) \bar{w}_i$ is given by


\begin{equation}


  D(\lambda) = \sum_{n=1}^N \sum_{i=1}^M x_i^{(n)}


   \left[ \frac{w_i^{(n)} - \bar{w}_i^{(n)}}{


            \lambda w_i^{(n)} + (1 - \lambda ) \bar{w}_i^{(n)}} 


   \right]


\end{equation}


Since the log likelihood is a convex function of $\lambda$, this derivative


allows the optimal value of $\lambda$ to be found by a simple


binary chop algorithm, viz.


\begin{verbatim}


     function FindLambdaOpt:


        if (D(0) <= 0) return 0;


        if (D(1) >= 0) return = 1;


        l=0; r=1;


        for (k=1; k<=maxStep; k++){


           m = (l+r)/2;


           if (D(m) == 0) return m;


           if (D(m) > 0) l=m; else r=m;


        }


        return m;


\end{verbatim}





\htool{HSmooth} is invoked in a similar way to \htool{HERest}.


For example, suppose that the directory \texttt{hmm2} contains a set of


accumulator files output by the first pass of \htool{HERest} running in parallel


mode using as source the HMM definitions listed in \texttt{hlist} and 


stored in \texttt{hmm1/HMMDefs}.  Then the command


\begin{verbatim}


    HSmooth -c 4 -w 2.0 -H hmm1/HMMDefs -M hmm2 hlist hmm2/*.acc


\end{verbatim}


would generate a new smoothed HMM set in \texttt{hmm2}.  Here the \texttt{-w}


option is used to set the minimum mixture component weight in any state to


twice the value of \texttt{MINMIX}\index{minmix@\texttt{MINMIX}}.  


The \texttt{-c} option sets the maximum


number of iterations of the binary chop procedure to be 4.











%%% Local Variables: 


%%% mode: latex


%%% TeX-master: "htkbook"


%%% End: 


